\documentclass[a4paper, 12pt]{article}
\usepackage{a4wide}
\usepackage{textcomp}
\usepackage{listings}
\usepackage{color}
\usepackage{graphicx}
\usepackage[backend=biber,style=numeric]{biblatex}
\addbibresource{bibliography.bib}

\definecolor{dkgreen}{rgb}{0,0.6,0}
\definecolor{gray}{rgb}{0.5,0.5,0.5}
\definecolor{mauve}{rgb}{0.58,0,0.82}

\lstset{frame=tb,
  language=Java,
  aboveskip=3mm,
  belowskip=3mm,
  showstringspaces=false,
  columns=flexible,
  basicstyle={\small\ttfamily},
  numbers=none,
  numberstyle=\tiny\color{gray},
  keywordstyle=\color{blue},
  commentstyle=\color{dkgreen},
  stringstyle=\color{mauve},
  breaklines=true,
  breakatwhitespace=true,
  tabsize=3
}

\begin{document}

    \title{A Graphical Programming Language Editor}
    \author{Candidate No. 198719}
    \date{November 3, 2020}
    \clearpage\maketitle
    \thispagestyle{empty}

    \newpage\clearpage\thispagestyle{empty}
    \tableofcontents
    \newpage
    \setcounter{page}{1}

    \section{Introduction}
    Graphical programming can be a fantastic and intuitive way to introduce new programmers to
    the scene. When programmers ask others for assistance, those helping typically do so in a visual 
    style, using whiteboards, drawing flowcharts, with boxes and arrows indicating the flow 
    of the program. Why can't we make programs in the same style if we find it so helpful to read? 
    The concept behind graphical programming is specifying the elements of the program graphically 
    rather than textually~\cite{dehouck2015maturity}.
    
        \subsection{Project Aim and Objectives}
    



    Project aim and objectives. (SMART) \\
    Needs of intended users. \\
    Brief description of the problem area. \\
    Motivation \\
    Project relevance. \\
    What will you do? \\ 
    How will you do it? \\ 
    Why will you do it? \\


    \section{Professional and Ethical Considerations}
    Discuss the ethical issues your prject is likely to deal with: \\
    - awareness of technological procedures and standards. \\
    - understanding and complying with legisaltion. \\
    If need humans, include plan and timetable for obtaining ethical approval.

    \section{Related Work}
    Scratch and App Inventor - I used app inventor for my GCSE assignment, making an application for a university campus.

    Do some background research. \\
    What has already been done in this field? \\
    How is your approach different and better than others? \\
    What is the novelty of your project, compared to others? \\
    What is your contribution to the field? \\
    Do not re-invent the wheel! \\

    \section{Requirements Analysis}
    Does your project meet the needs of a target group of users? \\
    What are the needs of these target users? \\
    How would an ideal system meet their needs? \\
    To what extent does your solution contribute to this? \\
    What do you expect to achieve within the given time? \\
    What will you definitely not achieve? \\
    You should stick to this analysis till the end of your project!

    \section{Project Plan}
    Describe what you have already done. \\
    Schedule what you still want to do. \\
    How are your tasks interdependent? \\
    Writing the draft report should be one of your main tasks.

    \section{Interim Log}
    Log of the meetings with your supervisor. \\
    Should cover discussions of every important project stage. \\
    Record the prupose and the outcome of every meetings. \\
    Note how these meetings helped you follow your plan. \\
    Include this log as an appendix to your report.
    
    \section{Appendices}
        \subsection{Project Proposal}

    \printbibliography
\end{document}